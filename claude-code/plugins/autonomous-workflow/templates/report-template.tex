% ABOUTME: Base LaTeX template for autonomous workflow research reports.
% ABOUTME: Provides standard document structure, packages, and section placeholders.

\documentclass[11pt]{article}
\usepackage[margin=1in]{geometry}
\usepackage{hyperref}
\usepackage{booktabs}
\usepackage{enumitem}
\usepackage{natbib}
\usepackage{graphicx}
\usepackage{longtable}
\usepackage{parskip}

\hypersetup{
  colorlinks=true,
  linkcolor=blue,
  citecolor=blue,
  urlcolor=blue
}

\title{PLACEHOLDER_TITLE}
\author{Research Report --- Generated via Autonomous Workflow}
\date{\today}

\begin{document}
\maketitle
\tableofcontents
\newpage

\section{Executive Summary}
% 1-2 page summary of all findings.
% This section is written last, after all research is complete.

\section{Background \& Context}
% Problem domain, why this matters, what motivated the research.

\section{Key Findings}
% Each subsection represents a thematic finding.
% Findings are organized thematically, not chronologically.
% Each finding has: claim, evidence, sources, confidence level.

\section{Analysis \& Synthesis}
% Cross-cutting analysis, patterns across findings.
% Logical conflicts identified and resolved.
% Novel insights from combining sources.

\section{Open Questions}
% What couldn't be determined from available sources.
% What would require primary research.
% Contradictions that remain unresolved.

\section{Methodology}
% How many iterations, how many sources consulted.
% Source credibility criteria used.
% Search strategies employed.

\bibliographystyle{plainnat}
\bibliography{sources}

\end{document}
